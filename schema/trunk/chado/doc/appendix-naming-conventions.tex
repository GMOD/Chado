\chapter{Chado Naming Conventions}

\section{Case sensitivity}

We use lowercase in all tables and column names - DBMSs differ in how
they treat case sensitivity. oracle will auto caps everything. so it's
best to be neutral and use lowercase.

\section{Table names}

In table names, we use underscores for linking tables; eg
\dbTable{feature\_dbxref} is a linking table between \dbTable{feature} and \dbTable{dbxref}

where a table name is a noun phrase rather than a single noun, we
concatenate the words together. for instance the table for describing
feature properties is called \dbTable{featureprop}. it could be argued
this is harder to read, but it does allow consistent usage of
underscores as above. FeatureProp could be used where it is known the
DBMS is case insensitive.

\section{Column names}

in column names, we also use concatenated noun phrases, except in the
case of primary / foreign keys, eg \dbColumn{dbxref\_id}.

we try to keep column names unique where appropriate, which is useful
for large join statements / views, in avoiding column name clash
between different tables. the convention is to use an abbreviated form
of the table name plus a noun describing the column, for instance
\dbColumn{fmin} in the \dbTable{feature}
table. by consistently using abbreviated forms we stop column names
getting too big [many DBMSs will barf on long column names]

\subsection{Primary and foreign key names}

we use the same column name for primary and foreign key columns - very
useful for NATURAL JOIN statements

