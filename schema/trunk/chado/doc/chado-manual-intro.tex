\chapter{Introduction}

general intro here

outlines - overview then module descriptions

See also \GMOD

and \FlyBase

\section{Schema}

naming convention

design patterns

\subsection{Module System}

\subsubsection{Module Metadata}

\section{View layers}

Views can be thought of as {\em virtual tables}. They provide a
powerful abstraction layer over the database.

Views in chado are defined on a per module basis. View definitions are
maintained in the {\tt chado/modules/MODULE-NAME/views}
directory.

Collections of view definitions are bundled into packages,
each package is a {\tt .sql} file.

\subsection{Inter-schema bridges}

\subsubsection{GODB Bridge}

\subsubsection{BioSQL Bridge}

\section{DBMS Functions}

DBMS Functions in \Chado are entirely optional

Functions in chado are defined on a per module basis. Function definitions are
maintained in the {\tt chado/modules/MODULE-NAME/functions} directory.

Collections of function definitions are bundled into packages. Each
package comes with an {\em interface descriptions} and one or more
{\em implementations}.

\subsubsection{Function Interface Definitions}

The interface descriptions are stored in a {\tt .sqlapi} file. The
syntax used is a variant of SQL and is intended primarily as a
consistent way of providing information for human, although it should
be parseable by software.

Here is an example, taken from the top of the {\tt
chado/modules/sequence/functions/subsequence.sqlapi} package. This
package provides basic subsequencing functions. It has dependencies on
two other function packages, declared at the top of the file. The
package declares multiple functions, only the first of which is show
here, a function for extracting subsequences from the sequence of a
feature.

\begin{verbatim}
IMPORT reverse_complement(TEXT) FROM 'sequtil';
IMPORT get_feature_relationship_type_id(TEXT) FROM 'sequence-cv-helper';

-----------------------------------
-- basic subsequencing functions --
-----------------------------------

DECLARE FUNCTION subsequence(
   srcfeature_id        INT REFERENCES feature(feature_id),
   fmin                 INT,
   fmax                 INT,
   strand               INT
)
 RETURNS TEXT;

COMMENT ON FUNCTION subsequence(INT,INT,INT,INT) IS 'extracts a
subsequence from a feature referenced by srcfeature_id, within the
interbase boundaries determined by fmin and fmax, reverse
complementing if strand = -1. The sequence can be DNA or AA. Strand
must always by >0 for AA sequences';
\end{verbatim}


\subsubsection{Function Implementations}

The goal is to provide implementations for different dialects of
procedural SQL. Currently only PostgreSQL dialect is supported. The
psql implementations are stored in {\tt .plpgsql} files.




\section{Software}
