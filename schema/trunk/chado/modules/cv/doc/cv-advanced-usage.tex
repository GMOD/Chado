\section{Chado and advanced ontology features}

This section describes advanced usage of the \cv module for use with
OWL-DL \cite{OWL}, advanced Obo format 1.2 \cite{OboFormat} features
or elements from other ontology formalisms.

If you aren't sure what this means, you probably don't need to read
this section yet.

\subsection{Background}

See the document on \cite{ConvertingOboToOWL}

\subsection{Logical definitions}

In a normal ontology DAG representation in chado, the
cvterm_relationship rows represent relationships between terms, or
more formally, {\em necessary conditions}. A logical definition must
have both {\em necessary and sufficient conditions}. A logical
definition often consists of a {\em generic term} (aka genus) and one
or more {\em discriminating characteristics} (aka differentiae). The
discriminating characteristics are typically relationships 

For example, the logical definition of {\tt larval locomotory
behaviour} would be a {\tt locomotory behaviour} (genus) which {\em
during} {\tt larval stage} (where {\em during} could be drawn from an
ontology of relations, and larval stage may come from an insect
developmental stage ontology). These constitute both necessary and
sufficient conditions: the conditions are necessary in that all
instances of larval locomotory behavior are necessarily locomotory
behaviors and are necessarily manifested at the larval stage. We could
represent this using a normal DAG. However, because this is a
definition it also constitutes sufficient conditions, in that any
instance of locomotory behavior which manifests at the larval stage is
by definition a larval locomotory behavior.

In an ontology formalism like OWL-DL or Obo-1.2, genus-differentiae
are represented using set-intersections.

Here is the Obo 1.2 representation:

\begin{verbatim}
[Term]
id: GO:0008345
name: larval locomotory behavior
namespace: biological_process
is_a: GO:0007626                      ! locomotory behavior
is_a: GO:0030537                      ! larval behavior
intersection_of: GO:0007626           ! GENUS: locomotory behavior
intersection_of: during FBdv:00005336 ! DIFFERENTIUM: during larval stage
\end{verbatim}

Here is the equivalent in OWL (note: RDF-XML syntax is very verbose!):

\begin{verbatim}
  <owl:Class rdf:ID="GO_0008345">
    <rdfs:label xml:lang="en">larval locomotory behavior</rdfs:label>
    <rdfs:subClassOf rdf:resource="#GO_0007626"/>
    <rdfs:subClassOf rdf:resource="#GO_0030537"/>
    <owl:equivalentClass>
      <owl:Class>
        <owl:intersectionOf rdf:parseType="Collection">
          <owl:Class rdf:about="#GO_0007626"/>
          <owl:Restriction>
            <owl:onProperty>
              <owl:ObjectProperty rdf:about="#during"/>
            </owl:onProperty>
            <owl:someValuesFrom rdf:resource="#FBdv_00005336"/>
          </owl:Restriction>
        </owl:intersectionOf>
      </owl:Class>
    </owl:equivalentClass>
  </owl:Class>
\end{verbatim}

When converting to chado we employ a more economical representation,
in terms of the number of triples we use:

\begin{verbatim}
  <!-- normal DAG relationships (necessary conditions) -->
  <cvterm_relationship>
    <type_id>is_a</type_id>
    <subject_id>GO:0008345</subject_id>
    <object_id>GO:0007626</object_id>
  </cvterm_relationship>
  <cvterm_relationship>
    <type_id>is_a</type_id>
    <subject_id>GO:0008345</subject_id>
    <object_id>GO:0030537</object_id>
  </cvterm_relationship>

  <!-- Genus/generic term -->
  <cvterm_relationship>
    <type_id>intersection_of</type_id>
    <subject_id>GO:0008345</subject_id>
    <object_id>GO:0007626</object_id> <!-- locomotory behavior -->
  </cvterm_relationship>

  <!-- Discriminating characteristics -->
  <cvterm_relationship>
    <type_id>intersection_of</type_id>
    <subject_id>GO:0008345</subject_id>
    <object_id>

      <!-- anonymous term representing  during(larval stage) -->
      <cvterm>
        <dbxref_id>
          <dbxref>
            <db_id>internal</db_id>
            <accession>restriction--OBOL:during--GO:0008345</accession>
          </dbxref>
        </dbxref_id>

        <!-- note: as this is an anon term, the name will never
             be shown to a user -->
        <name>restriction--OBOL:during--GO:0008345</name>
        <cv_id>anonymous_cv</cv_id>
        <cvtermprop>
          <type_id>is_anonymous</type_id>
          <value>1</value>
          <rank>0</rank>
        </cvtermprop>
        <cvterm_relationship>
          <type_id>OBOL:during</type_id>
          <object_id>FBdv:00005336</object_id>
        </cvterm_relationship>
      </cvterm>

    </object_id>
  </cvterm_relationship>

\end{verbatim}

(if you are not familiar with how ChadoXML maps to the chado schema,
see the explanation below)

If you wish to convert Obo-specified logical definitions to chadoxml
you will need go-perl v0.05 or higher (if you have a lower version,
the intersection\_of tags will simply be ignored).

\begin{verbatim}
go2chadoxml ont.obo > ont.chado
\end{verbatim}

\subsubsection{How logical definitions are stored in Chado}

This involves no schema changes to the cv module. Each intersection\_of
goes in as a DAG arc of type internal:intersection\_of. The object_id
in the arc is either a term (for the genus) or an anonymous term
representing a restriction (the differentium). the restriction has a
relationship of some type to another term.

For example, for "larval locomotory behavior" we would normally just have:

\begin{verbatim}
LLB is_a LocomotoryBehavior
LLB is_a LarvalBehavior
\begin{verbatim}

If we load a logical definition for this term (see
go-dev/go-perl/t/data/llm/obo), like this:

\begin{verbatim}
[Term]
id: GO:0008345
name: larval locomotory behavior
namespace: biological_process
is_a: GO:0007626                      ! locomotory behavior
is_a: GO:0030537                      ! larval behavior
intersection_of: GO:0007626           ! locomotory behavior
intersection_of: during FBdv:00005336 ! larval stage
\end{verbatim}

Then the intersection\_ofs get stored using the basic DAG tables as:

\begin{table}[htb]
\center
{ \small
\begin{tabular}{l l l}
Subject & Relation & Object \\ \hline
LLB & intersection\_of & LocomotoryBehavior
LLB & intersection\_of & anon:xxx
anon:xxx & during & FBv:00005336

\label{tab:intersections-in-Chado}
\end{tabular}
}
\caption{Logical definition stored ib cvterm\_relationship table}
\label{tab:tab-esc-str}
\end{table}

This uses 4 cvterm\_relationships and the creation of a new
``anonymous'' term that is never shown directly to the user. The
anonymous term represents the class of things that happen during the
larval stage

\subsubsection{Logical Definition Views}

Two views: cvterm\_genus and cvterm\_differentium views are in
chado/modules/cv/views 

\subsubsection{Example use case: Phenotypes}

The idea here is that queries for composed term "syndactyly" should
automatically return the same results as a boolean query for
"fusion"+inheres_in="finger" regardless of whether the annotation is
to the composed term or is a composed annotation (provided we put the
logical definition of syndactyly in the database)

\subsubsection{Example use case: feature types}

The Sequence Ontology has some logical definitions - you will need to
load the file {\tt so-xp.obo}

\subsubsection{Example use case: GO}

See
http://www.fruitfly.org/~cjm/obol

\subsubsection{Example use case: Drawing DAGs}

Currently the DAGs of many OBO ontologies are highly tangled; see:
http://www.fruitfly.org/~cjm/obol/doc/go-complexity.html

If all terms have logical definitions, then there is only one 'true'
(genus) \isa parent. This enables us to disentangle the DAGs and draw
distinct hierarchies. For example, the GO term {\em cysteine
biosynthesis} could be drawn as two distinct hierarchies - one process
and one chemical
