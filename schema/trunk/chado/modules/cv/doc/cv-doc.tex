\chapter{The cv Module: Ontologies}

\section{Introduction}

We have seen how the sequence module makes extensive use of terms
taken from various ontologies such as SO and the OBO Relations
Ontology, using the type\_id foreign key column. In addition, features
can be annotated using ontologies such as GO using the feature\_cvterm
linking table. These terms are modelled using the cv module, the core
of which is the cvterm table.

An ontology, terminology or cv (controlled vocabulary) , is a
collection of terms (here equivalent to what are more typically called
classes, types, categories or kinds in the ontology literature[REF])
in a particular domain of interest. Examples include "gene" (from SO),
"transcription factor activity" (from GO molecular function) and
"lymphocyte" (from OBO-Cell).  The chado cv module is based on the GO
Database schema, described here[14].  Terms are stored in the cvterm
table, and relationships between terms are stored in the
cvterm\_relationship table. This table follows an analogous structure
to the feature\_relationship table, in that it has columns subject\_id,
object\_id and type\_id. Here, all three of these foreign keys refer to
rows in the cvterm table.

A detailed treatment of relationship types in biological ontologies
can be found here[13]. Of particular interest to Chado is the is\_a
relation, which specifies a sub- typing relationship between two terms
or classes. Recall that tables in the sequence module frequently (such
as the feature table) defined a type\_id foreign key column to indicate
the specific type or class of entity for each row in that table. The
combination of the type\_id column and the is\_a relationship gives
Chado a data sub- classing system, beyond what is possible with
traditional SQL database semantics.

This is discussed further in a later section The collection of
cvterms and cvterm\_relationships can be considered to constitute
vertices and edges in a graph. This graph is typically acyclic (a
DAG), though it is not guaranteed to be as certain relationship types
are allowed to form cycles.

\subsection{Transitive Closure}

\subsubsection{Rules}

The cvtermpath is for calculating the reflexive transitive closure of
a relationship, and any derived relationships

Normal (direct) relationships are stored in the cvterm\_relationship
table. A entry in this table represents a cvterm\_relationship S over
some relation R.

\begin{verbatim}
 S = Subj R Obj
\end{verbatim}

For example:

 {\tt S = "cardioblast" develops\_from "mesodermal cell"}

The relation \isa represents a special kind of relation -
subsumption, or inheritance.

If X \isa Y, then it follows that all of Y's cvterm\_relationship
statements are inherited by X

\begin{verbatim}
[Rule 1]
If   X is_a Y
and  Y R Z
then X R(inh) Z
\begin{verbatim}

For example

\begin{verbatim}
     "cilium axoneme"  is_a    "axoneme"
     "axoneme"         part_of "cell projection"
THEREFORE:
     "cilium axoneme"  part_of(inh) "cell projection"
\end{verbatim}

Here we use T(inh) to represent an inherited relationship.

\subsubsection{Populating cvterm\_path}

The cvtermpath table stores the reflexive transitive closure of a
relationship, taking into account subsumption/inheritance. The number
of intermediate relationships is represented in the 'distance' column
of the table.

Here we use T(path) to represent the 'path' or closure of a
relationship. Every T(path) is stored in cvtermpath. We use the same
cvterm for T, the fact that it is a path is implicit.

We use these rules:

Reflexive relationships:

for all relations T,
  X T(path) X 

In this case the distance=0

Direct relationships:

these are also included in the cvtermpath table, distance=1

 If   X T       Y
 Then X T(path) Y

Transitive relationships:

these have distance > 1; these also make use of inheritance rule,
[Rule1], which gives us T(inh)

If   X T(inh)     Y
and  Y T(path)    Z
Then X T(path)    Z

Note that this rule is recursive.

These rules should be used for populating cvtermpath. Attempting to
calculate a more general closure where all relations are
treated equally or ignored will produce combinatorial explosions over
certain ontologies (eg flybase anatomy ontology)

What does this mean in practice?

For a typical database, which may only have relations \isa,
part\_of and develops\_from, we will end up with 3 sets of paths.

The \isa closure, \isa (path) will include paths over
cvterm\_relationships that look like this:

\begin{verbatim}

a is_a b is_a c is_a d is_a e

The "part_of" closure, part_of(path) will include paths over
cvterm_relationships that look like this:

a is_a b part_of c part_of d is_a e part_of f

The "develops_from" closure, develops_from(path) will include paths over
cvterm_relationships that look like this:

a develops_from b develops_from c is_a d is_a e develops_from f

\end{verbatim}

It may be tempting to mix different non \isa relationships in the same
path, but this should NEVER be done - there will be an unacceptable
combinatorial explosion in many cases. Besides, there is no use for
such a cvtermpath; it is meaningless.

Note that for amigolike query behaviour, it is necessary only to query
cvtermpath ignoring cvtermpath.type\_id (these are obtained by querying
cvterm\_relationship)



