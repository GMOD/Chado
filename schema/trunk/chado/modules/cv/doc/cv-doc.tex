\chapter{The cv Module: Ontologies}

\section{Introduction}

We have seen how the sequence module makes extensive use of terms
taken from various ontologies such as SO and the OBO Relations
Ontology, using the type\_id foreign key column. In addition, features
can be annotated using ontologies such as GO using the feature\_cvterm
linking table. These terms are modelled using the cv module, the core
of which is the cvterm table.

An ontology, terminology or cv (controlled vocabulary) , is a
collection of terms (here equivalent to what are more typically called
classes, types, categories or kinds in the ontology literature[REF])
in a particular domain of interest. Examples include "gene" (from SO),
"transcription factor activity" (from GO molecular function) and
"lymphocyte" (from OBO-Cell).  The chado cv module is based on the GO
Database schema, described here[14].  Terms are stored in the cvterm
table, and relationships between terms are stored in the
cvterm\_relationship table. This table follows an analogous structure
to the feature\_relationship table, in that it has columns subject\_id,
object\_id and type\_id. Here, all three of these foreign keys refer to
rows in the cvterm table.

A detailed treatment of relationship types in biological ontologies
can be found here[13]. Of particular interest to Chado is the is\_a
relation, which specifies a sub- typing relationship between two terms
or classes. Recall that tables in the sequence module frequently (such
as the feature table) defined a type\_id foreign key column to indicate
the specific type or class of entity for each row in that table. The
combination of the type\_id column and the is\_a relationship gives
Chado a data sub- classing system, beyond what is possible with
traditional SQL database semantics.

This is discussed further in a later section The collection of
cvterms and cvterm\_relationships can be considered to constitute
vertices and edges in a graph. This graph is typically acyclic (a
DAG), though it is not guaranteed to be as certain relationship types
are allowed to form cycles.

\section{Transitive Closure}


